\chapter{Mikrofonverstärker}
\section{Berechnung und Dimensionierun}
\subsection{Offsetspannung}
\subsubsection{Fragestellung: Was versteht man unter Offsetspannung und wie kommt sie zustande?}

Unter der Offsetspannung versteht man jene Spannung, die zwischen den beiden Eingängen des OPVs angelegt werden muss um eine Ausgangsspannung von \SI{0}{\volt} zu erhalten. Bzw. muss je nach verwendeter Definition $-U_{OS}$ zwischen den Eingängen angelegt werden.\\

Diese Spannung kommt durch Unsymmetrien zwischen den Transistoren der Eingangsdifferenzverstärkerstufe zustande.

\subsubsection{Bestimmen Sie mittels einer geeigneten Eingangsbeschaltung die Offsetspannung des Operationsverstärkers. Welches Ergebnis erwarten Sie?}

Erwartetes Ergebnis: Da es in der Simulation keine Schwierigkeit sein sollte ideale Transistoren zu modelieren, ist eine sehr geringe bis verschwindende Offsetspannung zu erwarten.

\textbf{Simulation:}
Zur Simulation wurde eine DC-Sweep-Simulation durchgeführt. Konkret wurde die Quelle $V_{IN}$ variiert und die stationäre Ausgangsspannung des OPVs betrachtet. Die Eingangsdifferenzspannung bei der Ausgang genau \SI{1.65}{\volt} beträgt wird als Offsetspannung definiert.
\begin{figure}[H]
    \centering
    \includegraphics[width = \textwidth]{Bilder/Schaltung_1.png}
    \caption{verwendete Schaltung und Simulationsparameter}
    \label{fig:my_label}
\end{figure}

\textbf{Simulationsergebnis:} Es wurde eine Offsetspannung von \SI{4.68}{\micro \volt} beobachtet.

\subsubsection{Wie kann die Offsetspannung in einer Simulation berücksichtigt werden?}
Die Offsetspannung könnte einfach durch eine zusätzliche Spannungsquelle, die genau die Offsetspannung liefert, an einem der beiden Eingänge des Operationsverstärkers modelliert werden.

\subsection{Bode-Plot}
\subsubsection{Simulieren Sie mit einer geeigneten Analyseart und Beschaltung den Amplituden- und Phasengang des Operationsverstärkers.}

Für diese Fragestellung eignet sich die AC-Sweep-Simulation. Hierfür wird die $V_{IN}$ zu einer Wechselspannungsquelle mit Spannungsamplitude \SI{1}{\volt} konfiguriert. In der AC-Simulation wird um den Arbeitspunkt linearisiert, eventuelle Austeuergrenzen und Sättigung spielen bei dieser Simulation keine Rolle. Simuliert wurde ein Frequenzbereich von \SI{1}{Hz} bis \SI{10}{\mega \hertz}.

\begin{figure}[H]
    \centering
    \includegraphics[width = \textwidth]{Bilder/Schaltung_2.png}
    \caption{verwendete Schaltung und Simulationsparameter}
    \label{fig:my_label}
\end{figure}

\textbf{Simulationsergebnis:} Das resultierende Bode-Diagram:
\begin{figure}[H]
    \centering
    \includegraphics[width = \textwidth]{Bilder/BodePlot.pdf}
    \caption{Frequenzgang}
    \label{fig:my_label}
\end{figure}

\subsubsection{Welche Verstärkung $A_0$ besitzt der Verstärker?}

Der Operationsverstärker besitzt eine Open-Loop-Verstärkung $A_0 = \frac{U_{OUT}}{U_{INP}-U_{INN}}$ von \num{3.74e4} beziehungsweise \SI{91.4}{dB}.

\subsubsection{Beeinflusst die Offsetspannung den Frequenzgang?}

Nein, die Offsetspannung hat keinen Einfluss auf den Frequenzgang, da die Schaltung für die Berechnung des Frequenzganges um den Arbeitspunkt linearisiert wird, es wird quasi mit dem Kleinsignalersatzschaltbild gerechnet.

\subsubsection{Welches GBW besitzt der OPV?}

Der Amplitudengang durchschreitet die \SI{0}{dB}-Linie bei einer Frequenz von \SI{1.13}{\mega \hertz}, dies entspricht auch dem Gain-Bandwith-Product.

\subsubsection{Wie kann die Verstärkung $A_0$ des OPV verändert werden?}

Die Open-Loop-Verstärkung $A_0$ ist eine Eigenschaft des OPVs und kann nicht durch externe Beschaltung verändert werden. Diese Verstärkung wird im wesentlichen durch die Spannungs-Verstärkungs-Stufe des OPVs bestimmt.

\subsection{Slew-Rate}

\subsubsection{Bestimmen Sie die Slew-Rate des OPV mit geeigneter Beschaltung.}

Zur Bestimmung der Slew-Rate wird der OPV als Impedanzwandler oder Spannungsfolger verschaltet, der Ausgang wird also auf den invertierenden Eingang rückgeführt. Als Eingangssignal wird eine Folge von Rechteckimpulsen aufgeschalten. Die Slew-Rate ist dann die Geschwindigkeit in der das Ausgangssignal diesem Impuls folgt.

\begin{figure}[H]
    \centering
    \includegraphics[width = \textwidth]{Bilder/Schaltung_3.png}
    \caption{Bufferschaltung zur Bestimmung der Slew-Rate}
    \label{fig:my_label}
\end{figure}

\textbf{Simulationsergebnis:} Es ergab sich folgender zeitlicher Verlauf der Ausgangsspannung und folgende Werte für die Slew-rate:

\begin{table}[H]
    \centering
    \begin{tabular}{|c|c|}
    \hline
         steigende Flanke & fallende Flanke  \\ \hline
         \SI{741}{\kilo \volt \per \second} & \SI{-5.11}{\mega \volt \per \second} \\ \hline
    \end{tabular}
    \caption{Caption}
    \label{tab:my_label}
\end{table}
\begin{figure}[H]
    \centering
    \includegraphics[width = \textwidth]{Bilder/Slewrate_default.pdf}
    \caption{Zeitlicher Verlauf der Spannungen}
\end{figure}

\subsubsection{Warum unterscheidet sich die Slew-Rate für die fallende und steigende Flanke}

Die Ursache für den Unterschied zwischen der Slew-Rate bei steigender und fallender Flanke ist vermutlich in der Ausgangsstufe des OPVs zu suchen. Bei steigender Flanke muss die Kapazität erst geladen werden, dies erfordert kurzzeitig hohe Ströme und hohe Leistungen. $\rightarrow$ Bei der steigenden Flanke ist man durch den maximalen Ausgangsstrom des OPV limitiert, bei der fallenden Flanke hingegen ist der maximale Sink-Current entscheidend, der typischerweise um einiges größer ist.

\subsubsection{Hat die Last einen Einfluss auf die Slew-Rate?}

Die Last hat einen sehr großen Einfluss auf die Slew-Rate. Ein größerer Kondensator würde z.B. mehr Ladung brauchen um auf die selbe Spannung geladen zu werden, um dies in der gleichen Zeit zu schaffen also mehr Leistung.\\

\textbf{Simulation mit der Last $C_1 = \SI{1}{\nano \farad}, R_3 = \SI{10}{\mega \ohm}$:}

\begin{table}[H]
    \centering
    \begin{tabular}{|c|c|}
    \hline
         steigende Flanke & fallende Flanke  \\ \hline
         \SI{33.3}{\kilo \volt \per \second} & \SI{-961}{\kilo \volt \per \second} \\ \hline
    \end{tabular}
    \caption{Caption}
    \label{tab:my_label}
\end{table}

\begin{figure}[H]
    \centering
    \includegraphics[width = \textwidth]{Bilder/Slewrate_1n.pdf}
    \caption{Zeitlicher Verlauf der Spannungen mit veränderter Last}
\end{figure}

Bei kleinerer Last lässt sich hingegen eine schnellere Slew-Rate beobachten. Auffällig ist auch, dass sich die Geschwindigkeiten für steigende und fallende Flanken annähern:

\textbf{Simulation mit der Last $C_1 = \SI{1}{\pico \farad}, R_3 = \SI{20}{\mega \ohm}$:}

\begin{table}[H]
    \centering
    \begin{tabular}{|c|c|}
    \hline
         steigende Flanke & fallende Flanke  \\ \hline
         \SI{5.55}{\mega \volt \per \second} & \SI{-5.47}{\mega \volt \per \second} \\ \hline
    \end{tabular}
    \caption{Caption}
    \label{tab:my_label}
\end{table}

\begin{figure}[H]
    \centering
    \includegraphics[width = \textwidth]{Bilder/Slewrate_1p.pdf}
    \caption{Zeitlicher Verlauf der Spannungen mit veränderter Last}
\end{figure}

\section{Simulation in LTSpcie}

\section{Anhang:}

Die Simulationen wurden mit LTSpice XVII durchgeführt. Die Diagramme wurden durch Datenexport von LTSpice mit Matplotlib erstellt.