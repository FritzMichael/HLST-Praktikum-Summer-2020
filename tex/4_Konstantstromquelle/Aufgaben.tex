\def\figpath{tex/4_Konstantstromquelle/pictures}
\graphicspath{{tex/4_Konstantstromquelle/pictures/}}

\chapter{Konstantstromquelle}
In Kapitel 4 wird die Auslegung einer Konstantstromquelle, welche mittels Bipolartransistoren realisiert wird, behandelt. Im folgenden wird auf die Berechnung, die Simulation und das Temperaturverhalten der Schaltung eingegangen. 

\section{Wahl der Stromquellenschaltung und Auslegung der Widerstände}
Für die zu untersuchende Stromquelle wird die Schaltung eines einfachen Stromspiegels gewählt, siehe Abb. \ref{fig_Kap4_01:Stromspiegel01}. 

\begin{figure}[H]
	\centering
	\def\svgwidth{0.6\textwidth}
	\input{\figpath/EinfacherStromspiegel.pdf_tex}
	\caption{Einfache Stromspiegelschaltung} 
	\label{fig_Kap4_01:Stromspiegel01} 
\end{figure}



\section{Maximaler Lastwiderstand $R_{L,max}$}

\section{Strom in Abhängigkeit des Lastwiderstandes $I_a(R_L)$}

\section{Verlauf des Innenwiderstandes $R_i$}

\section{Temperaturabhängigkeit des Ausgangsstroms bei globaler Temperaturänderung}

\section{Temperaturabhängigkeit der Schaltung bei unterschiedlichen Bauteiltemperaturen}

\section{Vergleich mit Berechnung}