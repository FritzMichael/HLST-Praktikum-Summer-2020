\def\figpath{tex/4_Konstantstromquelle/pictures}
\graphicspath{{tex/4_Konstantstromquelle/pictures/}}

\chapter{Konstantstromquelle}
In Kapitel 4 wird die Auslegung einer Konstantstromquelle, welche mittels Bipolartransistoren realisiert wird, behandelt. Im folgenden wird auf die Berechnung, die Simulation und das Temperaturverhalten der Schaltung eingegangen. 

\section{Wahl der Stromquellenschaltung und Auslegung der Widerstände}
Für die zu untersuchende Stromquelle wird die Schaltung eines einfachen Stromspiegels gewählt, siehe Abb. \ref{fig_Kap4_01:Stromspiegel01}. 

\begin{figure}[H]
	\centering
	\def\svgwidth{0.5\textwidth}
	\input{\figpath/EinfacherStromspiegel.pdf_tex}
	\caption{Einfache Stromspiegelschaltung} 
	\label{fig_Kap4_01:Stromspiegel01} 
\end{figure}

Die Transistoren $T_1$ und $T_2$ werden baugleich gewählt (jeweils Typ BC547B), womit auch 
\begin{equation}
    B_1 = B_2 = B
\end{equation}
gelten soll. 

Des weiteren gilt für den Kollektorstrom an einem npn-Bipolartransistor

\begin{equation}
    \label{glgn_transist}
    I_C = I_S \left(e^{\frac{U_{BE}}{U_T}} - 1 \right) .
\end{equation}

Für Masche I aus Abb. \ref{fig_Kap4_01:Stromspiegel01} gilt

\begin{equation}
    \text{MI: } U_{BE,1} + I_{C,1}\left( 1 + \frac{1}{B}\right)R_{E,1} = U_{BE,2} + I_{C,2}\left( 1 + \frac{1}{B}\right)R_{E,2} .
\end{equation}

Wählt man beide Emitterwiderstände gleich groß (jedoch $\nequ 0 $), so ergibt sich die Gleichung

\begin{equation}
    U_{BE,1} + I_S \left(e^{\frac{U_{BE,1}}{U_T}} - 1 \right)\left( 1 + \frac{1}{B}\right)R_E = U_{BE,2} + I_S \left(e^{\frac{U_{BE,2}}{U_T}} - 1 \right)\left( 1 + \frac{1}{B}\right)R_E .
    \label{glng_transzend}
\end{equation}

Es ist naheliegend, dass sich durch den symmetrischen Schaltungsaufbau als Lösung der transzendenten Gleichung \ref{glng_transzend} wiederum gleich große Basis-Emitterspannungen einstellen werden, 

\begin{equation}
    U_{BE,1} = U_{BE,2} .
\end{equation}

Aus Gleichung \ref{glgn_transist} folg somit
\begin{equation}
    I_{C,1} = I_{C,2} = I_C \quad \Rightarrow \quad I_{B,1} = I_{B,2} = I_B
\end{equation}

Mithilfe von Knoten I erhält man die Gleichung

\begin{equation}
    I_e = I_C + 2 \cdot I_B = I_C \cdot \left( 1 + \frac{2}{B} \right) .
\end{equation}

Daraus lässt sich ein Zusammenhang zwischen Referenzstrom $I_e$ und Ausgangsstrom $I_a$ finden, 

\begin{equation}
    I_C = I_a = \frac{1}{1 + \frac{2}{B}} \cdot I_e  = k_I \cdot I_e .
\end{equation}

Aus Masche II ergibt sich folgender Zusammenhang

\begin{equation}
    U_B = U_{BE} + I_e \cdot R_1 + I_a \cdot \left(1 + \frac{1}{B} \right) \cdot R_E = U_{BE} + I_e \cdot \left( R_1 + k_I \left(1 + \frac{1}{B} \right) \cdot R_E \right) .
\end{equation}

Der Ausgangsstrom lässt sich daher folgendermaßen ausdrücken

\begin{equation}
    I_a = k_I I_e = k_I \frac{U_B - U_{BE}}{R_1 + k_I \left(1 + \frac{1}{B} \right) \cdot R_E} .
\end{equation}

Will man nun einen Ausgangsstrom von 

\begin{equation*}
    I_a = \SI{10}{\milli\ampere}
\end{equation*}

bei gegebenen Bauteilwerten (siehe Tab. \ref{tab_Kap4_01:Bauteilwerte} ) so ergibt sich für den Widerstand $R_1$

\begin{equation}
    R_1 = k_I \cdot \left(\frac{U_B - U_{BE}}{I_a} - R_1 \cdot \left( \right)\right) = \frac{1}{1+\frac{2}{290}}\left(\frac{\SI{15}{\volt} - \SI{0.7}{\volt}}{\SI{10}{\milli\ampere}} - \SI{100}{\ohm} \cdot \left( 1 + \frac{1}{290} \right)\right) = \SI{1.32}{\kilo\ohm} .
\end{equation}

\begin{table}[H]
\centering
\begin{tabular}{|c|c|} \hline
Benennung & Größe \\ \hline
$U_B$ & \SI{15}{\volt} \\ \hline
$B$ & 290 \\ \hline
$R_E$ & \SI{100}{\ohm} \\ \hline
\end{tabular}
\caption{Parameter für Berechnung und Simulation}
\label{tab_Kap4_01:Bauteilwerte} 
\end{table}

Die Gleichstromverstärkung $B$ wurde dem Datenblatt bei $I_C = \SI{2}{\milli\ampere}$ und $U_{CE} = \SI{5}{\volt}$ entnommen.

Von einer Konstantstromquelle darf eine gewisse Präzision erwartet werden, womit der Widerstand entsprechend der E48-Reihe mit

\begin{equation*}
    R_1 = \SI{1.33}{\kilo\ohm}
\end{equation*}

dimensioniert werden kann.

\section{Maximaler Lastwiderstand $R_{L,max}$}
Aus dem Datenblatt des BC547B lässt sich bei $I_C = \SI{10}{\milli\ampere}$ eine Kollektor-Emitter-Sättigungsspannung von 

\begin{equation}
    U_{CE,sat} = \SI{0.09}{\volt}
\end{equation}

ablesen.

Legt man eine Masche über die Ausgangsseite, ergibt sich

\begin{equation}
    U_B = R_{L,max} \cdot I_a + U_{CE,sat} + R_E \cdot I_a \cdot \left(1 + \frac{1}{B} \right) .
\end{equation}

Somit beträgt der maximale Lastwiderstand, ab der der Kollektorstrom einbricht

\begin{equation}
    R_{L,max} = \frac{U_B - U_{CE,sat}}{I_a} - R_E = \frac{\SI{15}{\volt} - \SI{0.09}{\volt}}{\SI{10}{\milli\ampere}} - \SI{100}{\ohm} = \SI{1.39}{\kilo\ohm} .
\end{equation}

\section{Strom in Abhängigkeit des Lastwiderstandes $I_a(R_L)$}
Für die LTSpice-Simulation wurde das Schematic aus Abb. \ref{fig_Kap4_02:SpiceSchematic} verwendet. 

\begin{figure}[H]
    \centering
    \includegraphics[width = 0.6\textwidth]{\figpath/einfacherStromspiegel.jpg}
    \caption{Einfacher Stromspiegel in LTSpice}
    \label{fig_Kap4_02:SpiceSchematic}
\end{figure}

Das Simulationsergebnis kann in Abb. \ref{fig_Kap4_03:Ia} betrachtet werden.

\begin{figure}[H]
	\centering \small
	\scalebox{0.9}{% This file was created by matlab2tikz.
%
\definecolor{mycolor1}{rgb}{0.00000,0.44700,0.74100}%
%
\begin{tikzpicture}

\begin{axis}[%
width=4.521in,
height=3.566in,
at={(0.758in,0.481in)},
scale only axis,
xmin=0,
xmax=10000,
xlabel style={font=\color{white!15!black}},
xlabel={$R_L \text{ in } \Omega$},
ymin=0.001,
ymax=0.011,
ylabel style={font=\color{white!15!black}},
ylabel={$I \text{ in mA}$},
axis background/.style={fill=white},
title style={font=\bfseries},
title={$I_a(R_L)$},
xmajorgrids,
ymajorgrids
]
\addplot [color=mycolor1, forget plot]
  table[row sep=crcr]{%
10	0.0102021\\
110	0.01019756\\
210	0.01019295\\
310	0.01018826\\
410	0.01018349\\
510	0.01017864\\
610	0.01017371\\
710	0.01016869\\
810	0.01016358\\
910	0.01015837\\
1010	0.01015307\\
1110	0.01014766\\
1210	0.01014216\\
1310	0.01013654\\
1410	0.009866105\\
1510	0.009252984\\
1610	0.008702559\\
1710	0.008211943\\
1810	0.007772927\\
1910	0.007378082\\
2010	0.007021202\\
2110	0.006697128\\
2210	0.00640157\\
2310	0.006130943\\
2410	0.005882232\\
2510	0.005652887\\
2610	0.005440736\\
2710	0.005243917\\
2810	0.005060831\\
2910	0.00489009\\
3010	0.004730487\\
3110	0.004580967\\
3210	0.004440606\\
3310	0.004308587\\
3410	0.004184188\\
3510	0.004066769\\
3610	0.003955759\\
3710	0.003850646\\
3810	0.003750974\\
3910	0.00365633\\
4010	0.003566343\\
4110	0.003480679\\
4210	0.003399033\\
4310	0.003321129\\
4410	0.003246716\\
4510	0.003175563\\
4610	0.003107462\\
4710	0.00304222\\
4810	0.002979661\\
4910	0.002919623\\
5010	0.002861956\\
5110	0.002806523\\
5210	0.002753197\\
5310	0.002701858\\
5410	0.0026524\\
5510	0.002604719\\
5610	0.002558722\\
5710	0.002514322\\
5810	0.002471435\\
5910	0.002429988\\
6010	0.002389907\\
6110	0.002351127\\
6210	0.002313586\\
6310	0.002277225\\
6410	0.002241988\\
6510	0.002207826\\
6610	0.002174689\\
6710	0.002142532\\
6810	0.002111312\\
6910	0.002080988\\
7010	0.002051524\\
7110	0.002022882\\
7210	0.001995029\\
7310	0.001967932\\
7410	0.001941562\\
7510	0.001915889\\
7610	0.001890886\\
7710	0.001866527\\
7810	0.001842788\\
7910	0.001819645\\
8010	0.001797076\\
8110	0.00177506\\
8210	0.001753577\\
8310	0.001732607\\
8410	0.001712134\\
8510	0.001692138\\
8610	0.001672604\\
8710	0.001653516\\
8810	0.001634859\\
8910	0.001616618\\
9010	0.00159878\\
9110	0.001581331\\
9210	0.001564258\\
9310	0.001547551\\
9410	0.001531196\\
9510	0.001515184\\
9610	0.001499503\\
9710	0.001484143\\
9810	0.001469095\\
9910	0.001454349\\
10000	0.001441328\\
};
\end{axis}
\end{tikzpicture}%}
	\caption{Ausgangsstrom in Abhängigkeit des Lastwiderstandes}
	\label{fig_Kap4_03:Ia}
\end{figure}

\section{Verlauf des Innenwiderstandes $R_i$}
Lt. Abb. \ref{} wird nun am Ausgang der Lastwiderstand durch eine Spannungsquelle ersetzt. 

\section{Temperaturabhängigkeit des Ausgangsstroms bei globaler Temperaturänderung}

\section{Temperaturabhängigkeit der Schaltung bei unterschiedlichen Bauteiltemperaturen}

\section{Vergleich mit Berechnung}