\def\figpath{tex/5_Differenzverstaerker/pictures}
\graphicspath{{tex/5_Differenzverstaerker/pictures/}}

\chapter{Differenzverstärker}
In Kapitel 5 werden die Eigenschaften einer einfachen und einer erweiterten Differenzverstärkerschaltung untersucht.

\section{Einfacher Differenzverstärker}

\subsection{Dimensionierung der Kollektorwiderstände $R_C$}

Im ersten Schritt müssen die Kollektorwiderstände dermaßen dimensioniert werden, so dass die Kollektorpotentiale, im Arbeitspunkt $U_P = U_N = \SI{0}{\volt}$, auf \SI{7.5}{\volt} liegen. Die Emitterwiderstände sollen \SI{33}{\ohm} betragen.

Der Strom der Stromquelle von \SI{10}{\milli \ampere} teilt sich gleichmäßig auf die Beiden Kollektorströme ($I_{c1} = I_{c2} = I_c$ auf (Basisströme werden gerechtfertigterweise vernachlässigt). Der Strom durch die Kollektorwiderstände beträgt also jeweils \SI{5}{\milli \ampere} und an den beiden Widerständen soll \SI{7.5}{\volt} abfallen. Dies liefert über das ohmsche Gesetz den gesuchten Widerstandswert:

\begin{equation}
    V_c = \SI{7.5}{\volt} = \SI{15}{\volt} - R_c I_c \rightarrow Rc = \frac{\SI{7.5}{\volt} - \SI{15}{\volt}}{- I_c} = \SI{1.5}{\kilo \ohm}
\end{equation}

\subsection{Kleinsignal-Spannungsverstärkung $A_{ed}$}

In diesem Punkt soll die Kleinsignal.Spannungsverstärkung für die zwei verschiedenen Werte $R_E = \SI{0}{\ohm}$ und $R_E = \SI{33}{\ohm}$ berechnet werden. Am Arbeitspunkt, also am Wert von $r_{BE}$ sollte die Änderung des Emitterwiderstandes nichts verändern, da der Kollektorstrom in dieser Schaltung von einer Stromquelle getrieben wird und unabhängig vom Kollektorwiderstand ist.

Zur Berechung der gesuchten Größe $A_{ed} = \frac{u_a}{u_p - u_n}$ wird das Kleinsignalersatzschaltbild analysiert:

FALSCHER ANSATZ: Werd das Thema morgen (Samstag) vormittag ausbessern

\subsubsection{$u_a$ zufolge $u_p$}

\begin{align}
    u_p =& I_{B,1} r_{BE} + I_E 2 R_E + I_{B,2} r_{BE} \\
    I_{B,1} (B+1) =& I_E \\
    I_{B,2} (B+1) =& I_E \\
    u_a = r_{BE} I_{B,2}
\end{align}

Aus den beiden Knotengleichungen folgt, dass $I_{B,1} = I_{B,2} = I_B$ gelten muss:

\begin{align}
    u_p =& I_{B} \left( 2 r_{BE} + (B+1) 2 R_E \right) \rightarrow I_B = \frac{u_p}{2 \left( r_{BE} + (B+1) R_E \right)}\\
    u_a =& r_{BE} I_{B} = u_p \frac{r_{BE}}{2 \left( r_{BE} + (B+1) R_E \right)}
\end{align}

\subsubsection{$u_a$ zufolge $u_n$}

\begin{align}
    u_n =& I_{B,2} r_{BE} + I_E 2 R_E + I_{B,1} r_{BE} \\
    I_{B,2} (B+1) =& I_E \\
    I_{B,1} (B+1) =& I_E \\
    u_a = - r_{BE} I_{B,2} + u_n
\end{align}

Aus den beiden Knotengleichungen folgt, dass $I_{B,1} = I_{B,2} = I_B$ gelten muss:

\begin{align}
    u_n =& I_{B} \left( 2 r_{BE} + (B+1) 2 R_E \right) \rightarrow I_B = \frac{u_n}{2 \left( r_{BE} + (B+1) R_E \right)}\\
    u_a =& - r_{BE} I_{B} + u_n = u_n \left( 1 - \frac{r_{BE}}{2 \left( r_{BE} + (B+1) R_E \right)} \right)
\end{align}

\subsubsection{$u_a$ zufolge beider Quellen}

\begin{equation}
    u_a = u_p \frac{r_{BE}}{2 \left( r_{BE} + (B+1) R_E \right)} + u_n \left( 1 - \frac{r_{BE}}{2 \left( r_{BE} + (B+1) R_E \right)} \right)
\end{equation}

Mit $u_n = - u_p$ eingesetzt:

\begin{align}
    u_a &= u_p \frac{r_{BE}}{2 \left( r_{BE} + (B+1) R_E \right)} + u_p \left( - 1 + \frac{r_{BE}}{2 \left( r_{BE} + (B+1) R_E \right)} \right)\\
    u_a &= u_p \left(\frac{r_{BE}}{\left( r_{BE} + (B+1) R_E \right)} - 1 \right) \\
    \frac{u_a}{u_p - u_n} = 
\end{align}

\subsection{Gleichtaktverstärkung $A_{gl}$}

\subsection{Gleichtaktverstärkung (Stromquelle durch Widerstand ersetzt)}

\subsection{Temperaturverhalten der Kollektorpotenziale}

\subsection{Auswirkungen einer globalen und einer Differenztemperaturänderung}

\subsection{Aussteuerbereich des Verstärkers}

\subsection{Überlagerung einer Brummspannung}

\subsection{Verhalten bei negativer Betriebsspannung}

\section{Erweiterter Differenzverstärker / Operationsverstärker}




