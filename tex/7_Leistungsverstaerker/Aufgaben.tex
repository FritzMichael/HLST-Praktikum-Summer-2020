\chapter{Leistungsendstufen}

In diesem Kapitel werden Leistungsendstufen mit zwei komplementären Transistoren behandelt.

\section{Übernahmeverzerrung}

\begin{figure}[H]
    \centering
    \includegraphics[width = \textwidth]{simulations/7_Leistungsendstufe/Übernahmeverzerrungen.pdf}
    \caption{Übernahmeverzerrungen}
    \label{fig:my_label}
\end{figure}

Trotz der Unterschiede in der Wellenform, waren für mich bei einer Grundfrequenz von \SI{1}{\hilo \hertz} keine Unterschiede hörbar.

Im Frequenzspektrum betrachtet, wirken sich diese nicht-Linearitäten  in der Verstärkung wie folgt aus:

\begin{figure}[H]
    \centering
    \includegraphics[width = \textwidth]{simulations/7_Leistungsendstufe/Übernahmeverzerrungen_fft.pdf}
    \caption{Übernahmeverzerrungen im Frequenzbereich}
\end{figure}

Man erkennt, dass besonders die ungeraden vielfachen der Grundfrequenz deutliche Oberwellen erzeugen (3, 5, 7 ... kHz).

\section{Gegentaktendstufe mit Ruhestromeinstellung}